\documentclass[landscape]{cheat}
\usepackage{amsmath}
\usepackage{amssymb}
\usepackage{listings}

\begin{document}
\begin{multicols*}{3}

\begin{center}
\Large{\underline{CSSE7610 Cheat Sheet}} \\
\end{center}

%% START HERE
\section{Mutual Exclusion and Verification}
\begin{description}
    \item[Mutual Exclusion] $\square \lnot(p1_\text{critical} \land p2_\text{critical})$.
        No two processes are simultaneously in their critical sections.
    \item[Freedom from Deadlock] $\square (p1_\text{attempt} \lor p2_\text{attempt}) \rightarrow
        \lozenge (p1_\text{critical} \lor p2_\text{critical})$.
        If some processes are attempting to enter the critical section, eventually one will succeed.
    \item[Freedom from Starvation] $\forall n \square \left(pn_\text{attempt} \rightarrow \lozenge pn_\text{critical}\right)$.
        If any process attempts to enter its critical section, eventually it will succeed.
\end{description}

\subsection{Fairness}
\begin{description}
    \item[Weak Fairness] If a scenario is continually enabled, it eventually appears in the scenario.
    \item[Strong Fairness] is a mystery
\end{description}

\subsection{Linear Temporal Logic}
\begin{description}
    \item[$\square$ always] conditions must be true at all time stamps.
    \item[$\lozenge$ eventually] conditions must be true at some future time stamp.
\end{description}

\section{Semaphores and Monitors}
\begin{description}
    \item[Semaphore] is a counter that can be incremented or decremented.
        It will block before going negative.
    \item[Monitor] is an object that can only be accessed (methods called on it) in one thread at a time.
        It will block before running a method in one thread if its busy in another.
    \item[Condition Variable] is a FIFO queue of sleeping processes.
        It has operations like \lstinline{wait}, \lstinline{notify} and \lstinline{notifyAll}.
\end{description}

\section{Locks and Data Structures}
\subsection{Lockless Techniques}
\begin{description}
    \item[Atomic References] support \lstinline{get}, and \lstinline{compareAndSwap} methods.
    \item[Compare and Swap] in a loop to update state iff nobody else has updated state since you started.
    \item[Back-offs] (exponential, random) in case of failed compare and swaps can ease load in case of contention.
    \item[Queues] are an alternative to random back offs that are deterministic and minimal in waiting time.
    \item[Cache Coherence] improves for other threads if you \lstinline{get} in a loop and ensure the value is correct before proceeding to \lstinline{getAndSet}.
\end{description}

\subsection{Locks}
\begin{description}
    \item[Fine grained] locking can improve rate at which concurrent threads access different parts of a data structure.
    \item[Optimistic] locking can speed things up if you are doing more reads than writes to a data structure.
\end{description}

\section{Distributed Algorithms}
\subsection{Problems}
\begin{itemize}
    \item Messages may take arbitrary time to be sent.
    \item Messages are not guaranteed to arrive in order.
\end{itemize}

\subsubsection{Ricart-Agrawala}
\begin{lstlisting}
myNum = 0
deferred = empty set
main loop
    myNum = chooseNumber
    for all other nodes N
        send(request, N, myID, myNum)
    await replies from all other nodes
    critical section
    for all nodes N in deferred
        remove N from deferred
        send(reply, N, myID)
receive loop
    receive(request, source, reqNum)
    if myNum < reqNum
        defer(source)
    else
        send(reply, N, myID)
\end{lstlisting}

\begin{description}
    \item[Inefficient] must contact all nodes every time.
    \item[Inefficient] does not get faster even without contention.
    \item[Starvation] \lstinline{chooseNumber} should choose a number that is larger than all other previously seen numbers.
\end{description}

\subsection{Neilsen-Mizuno}
\begin{lstlisting}
parent = {form a tree}
deferred = 0
holding = (parent == 0)
main loop
    if not holding
        send(request, parent, myID, myID)
        parent = 0
        receive(token)
    holding = false
    critical section
    if deferred != 0
        send(token, deferred)
        deferred = 0
    else
        holding = true
receive loop
    receive(request, source, originator)
    if parent == 0
        if holding
            send(token, originator)
            holding = false
        else
            deferred = originator
    else
        send(request, parent, myID, source)
    parent = source
\end{lstlisting}

\begin{description}
    \item[Tree] of nodes is maintained, with the root having the token.
    \item[Efficient] automatically flattens the tree as the token moves around.
    \item[Efficient] reduces the number of nodes that need to be contacted, especially in uncontested case.
\end{description}

\section{Real Time Sytems}

\subsection{Priority}
\begin{description}
    \item[Preemption] is when a high priority task steals the CPU from a lower one.
    \item[Priority inversion] is when a high priority task blocks waiting for a lower priority task to exit its critical section.
    \item[Priority inheritance] transfers the priority of the blocked task onto the blocking task for the duration of its critical section.
    \item[Priority ceiling locking] places priority on a monitor and lets tasks inherit that priority when locking the monitor.
\end{description}

\subsection{Process Scheduling}
\begin{description}
    \item[Deadline] when the process needs to be finished by.
        Normally relative to its starting time.
    \item[Period] how often the process starts.
    \item[Execution time] how long the process takes to run.
    \item[Release time] when the process is able to start (initial offset).
\end{description}

\subsection{Scheduling Algorithms}
\begin{itemize}
    \item Earliest deadline first.
    \item Rate monotonic (priority is inverse of execution time).
    \item Synchronous systems (calculated ahead of time with worst case execution times).
\end{itemize}

\subsection{Feasibility}
\begin{enumerate}
    \item Calculate the portion of time each process spends.
    \item Sum the portions.
    \item Feasibility occurs if the sum is less than 100\%.
\end{enumerate}

\end{multicols*}
\end{document}
