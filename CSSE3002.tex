% 1 Page Cheat Sheet
% Only the bare necessities go here

\documentclass[landscape]{cheat}

\begin{document}
\footnotesize
\begin{multicols}{3}

\begin{center}
\Large{\underline{CSSE3002 Cheat Sheet}} \\
\end{center}

%% START HERE
\section{Ethics and Professionalism}
Australian Computer Society (ACS)
\begin{description}
\item[Primary of Public Interest] place interest of public above personal, business or sectional interests.
\item[Enhancement of Quality of Life] Strive to enhance quality of life of those affected.
\item[Honesty] Honest representation of skills, knowledge, services and products.
\item[Competence] Work competently and diligently for all stakeholders.
\item[Professional Development] Enhance your own development and your colleagues and staff.
\item[Professionalism] Enhance integrity of the ACS and respect of members for each other.
\end{description}

\section{SE Process}
\subsection{Waterfall}
\begin{description}
    \item[Requirements] Define requirements.
    \item[Analysis] Analyse requirements, find and fix any inconsistencies and unclear areas.
    \item[Design] Create / plan the software architecture.
    \item[Coding] Develop, prove and integrate the software.
    \item[Testing] Systematically discover and debug defects, and prove the code is correct.
    \item[Operations] Install, migrate, support and maintain the completed system.
\end{description}

\subsection{Spiral}
\begin{description}
    \item[Identification] \textbf{Initially} gathering business requirements.
        \textbf{Subsequently} identification of system requirements, subsystem requirements and unit requirements.
    \item[Design] \textbf{Initially} conceptual design.
        \textbf{Subsequently} architectural design, logical design of modules, physical product design and the final design.
    \item[Construct or Build] \textbf{Initially} develop proof of concept model.
        \textbf{Subsequently} new release.
    \item[Evaluation and Risk Analysis] \textbf{Initially} Customer evaluation and feedback.
        \textbf{Subsequently} identify, estimate and monitor the technical feasibility and management risks, IE schedule slippage and cost overrun.
\end{description}

\subsection{Agile}
\begin{itemize}
    \item Requirements never fixed.
    \item Deliver early and deliver often.
\end{itemize}

\subsection{Lean}
\begin{itemize}
    \item Eliminate waste.
    \item Decide as late as possible.
    \item Deliver as fast as possible.
    \item See the whole.
    \item Think big, act small, fail fast.
\end{itemize}

\section{UX Design and RE}
\begin{description}
    \item[Useful] Original and fulfil a need.
    \item[Desirable] Evoke emotion and appreciation.
    \item[Accessible] Accessible to people with disabilities.
    \item[Credible] People must trust and believe.
    \item[Findable] Navigable.
    \item[Usable] Easy to use.
\end{description}

\section{Estimation}
\subsection{By Analogy}
\begin{enumerate}
    \item Characterize the proposed product.
    \item Select the most similar completed products whose characteristics have been stored in the historical database.
    \item Derive the estimate for the proposed project from the most similar completed projects by analogy.
\end{enumerate}
\subsubsection{Advantages}
\begin{itemize}
    \item Based on actual project characteristic data.
    \item The estimator's past experience and knowledge can be used which is not easy to be quantified.
    \item The differences between the completed and the proposed project can be identified and the impacts estimated.
\end{itemize}
\subsubsection{Disadvantages}
\begin{itemize}
    \item Restricted to the information that is available at that point.
    \item Difficult to find the right number of analogies to get an accurate estimation (go for 2).
\end{itemize}
Where applicable, this is superior to the algorithmic model in at least some circumstances.

\subsection{Wide-band Delphi (Expert Judgement)}
\begin{enumerate}
    \item Get multiple experts / stakeholders.
    \item Share project information.
    \item Each participant provides an estimate independently and anonymously.
    \item All estimates are shared and discussed.
    \item Repeat until consensus, or exclude extremes and calculate average.
\end{enumerate}
\subsubsection{Advantages}
\begin{itemize}
    \item People who would do the work are making the estimates, thus ensuring validity.
    \item Assumptions are documented, discussed and agreed.
    \item Simple.
\end{itemize}
\subsubsection{Disadvantages}
\begin{itemize}
    \item Management support is required.
    \item Management may not like estimation results.
\end{itemize}

\subsection{Parametric (Algorithmic) Models}
Based on (examples):
\begin{itemize}
    \item Research and historical data.
    \item Language, design methodology, skill-levels, risk assessments, etc.
\end{itemize}
\subsubsection{Advantages}
\begin{itemize}
    \item Repeatable estimations.
    \item Easy to modify input data, refine and customise formulas.
    \item Efficient and able to support a family of estimations or sensitivity analysis.
    \item Objectively calibrated to previous experiences.
\end{itemize}
\subsubsection{Disadvantages}
\begin{itemize}
    \item Unable to deal with exceptional conditions (personnel, teamwork, match between skill-levels and tasks).
    \item Poor sizing inputs and inaccurate cost driver rating will result in inaccurate estimation.
    \item Some experience and factors can not be easily quantified.
\end{itemize}
\subsubsection{Examples}
\begin{description}
    \item[COCOMO] Based on lines of code. Designed to model large, institutional projects. Includes overheads for design, management, etc.
    \item[Function Point Analysis] Based on the functional requirements of the system - number of inputs, outputs, database tables, etc.
\end{description}

\section{Testing}
\begin{description}
    \item[Validation] Demonstrate that the software meets its requirements.
    \item[Verification] Discover faults or defects in the software where its behaviour is incorrect or not in conformance with its specification.
\end{description}

\subsection{Development Testing}
\begin{description}
    \item[Unit Testing] Test units (methods or classes).
    \item[Integration / Component Testing] Test composite components, and component interfaces.
    \item[System] Test integration of whole system and component interactions.
\end{description}

\subsection{Release Testing}
\begin{itemize}
    \item Black box.
    \item Functionality, performance, dependability, reliability.
    \item Form of system testing.
\end{itemize}

\subsection{User Testing}
\begin{description}
    \item[Alpha] Users work with developers to test the software at the developer's site.
    \item[Beta] Experimental release to some users, problems raised with system developers.
    \item[Acceptance] Customers test to decide whether the product is ready for deployment.
\end{description}

\section{Requirements Engineering}
Acquisition, analysis, validation, documentation and management of requirements.

\subsection{Functional Requirements (use cases)}
Requirements or capabilities for functions (specific behaviour) that must be performed by the system.
\begin{itemize}
    \item Formalises expectations.
    \item Easy technique to understand.
    \item User-driven, so encourages user involvement.
    \item Basis for scoping and prioritizing development work.
    \item Basis for acceptance testing.
\end{itemize}

\subsubsection{Use cases need}
\begin{description}
    \item[Actor] The 'user' - note that the system \textit{can not} use itself.
    \item[Name] A verb phrase, from the primary actor's perspective.
    \item[Description] Summary of the use case.
    \item[Pre-conditions] That must be met before the use case can happen.
    \item[Steps] Sequence of user input / stimuli and system responses.
    \item[Alternative Sequences] Alternative paths of action, based on execution of use case.
    \item[Post-conditions] That must be met after the use case is complete.
    \item[A Package] Use case packages gather similar use cases together.
    \item[A priority] Must have, Should have, Could have, Won't have.
\end{description}

\subsubsection{Relationships between use cases}
\begin{description}
    \item[$<<$Include$>>$ relationship] Factors out common behaviour in use cases.
    \item[$<<$Extend$>>$ relationship] Factors out optional behaviour in use cases.
    \item[Generalize relationship] For when certain behaviour is overridden in child classes.
\end{description}

\begin{tabular}{|p{3.5cm}|p{3.5cm}|}
\hline
    \textsc{User Story} & \textsc{Use Case} \\ \hline
    Statement of user's needs. & Description of a single interaction with the system. \\ \hline
    Typically agile. & Typically iterative / traditional. \\ \hline
    User centered. & User centered. \\ \hline
\end{tabular}

\subsection{Non-functional Requirements Examples}
Constraints on performance or quality of the system.
\begin{description}
    \item[Safety Requirements] The system shall not permit operation unless the operator guard is in place.
    \item[Security Requirements] Only the system administrator can change system data.
    \item[Interface Requirements] The system's interaction wit other existing or proposed systems.
    \item[Human Engineering Requirements] (usability) adequacy, learning, error handling, recovering...
\end{description}

\subsection{Sources of requirements (stakeholders)}
\begin{description}
    \item[Users] customers or end users.
    \item[Other Stakeholders] marketing experts, regulators, managers, business owners, developers.
    \item[Non-Human Sources] other devices or systems in the environment.
\end{description}

\subsection{Requirements Elicitation}
\begin{description}
    \item[Interviews] Understanding problem and eliciting general requirements.
    \item[Workshops] Multiple stakeholders, resolve conflicting requirements, broad info about system usage.
    \item[Focus Groups] Broad representation, broad-based ideas.
    \item[Observations] Time consuming, and limited feedback from users.
    \item[Questionnaires] Inexpensive, easy to deploy. Hard to write.
    \item[System Interface Analysis] Look at the functionality of another system.
    \item[User Interface Analysis] Study existing systems, find things to replicate and avoid.
    \item[Document Analysis] Study documentation of existing systems, or industry standards / legislation.
\end{description}

\subsection{Classes}
\begin{itemize}
    \item High level structure of the system internals.
    \item Collected into packages.
    \item Represented in a class diagram, which displays logical / static structure of a system.
\end{itemize}

\subsubsection{Classes need}
\begin{description}
    \item[Name] Unique within its namespace.
    \item[Attributes] The information / state stored by instances of the class.
    \item[Operations] The methods / functional responsibilities of the class.
\end{description}

\subsubsection{Types of classes}
\begin{description}
    \item[Boundary Classes] separate interfaces from the system, and communicate with the environment.
    \item[Entity Classes] functionality dealing with storage and handling of long-lived / persistent information.
    \item[Control Classes] Control interaction between groups of objects. Only meant to handle one (or a few) use cases.
\end{description}

\subsubsection{Relationships between classes}
\begin{description}
    \item[Association] \textit{straight line} Covers any logical connection or relationship between classes.
    \item[Directed Association] \textit{arrow} Depicts a container-contained directional flow.
    \item[Reflexive Association] \textit{line from self to self} Relationships such as management of employees.
    \item[Multiplicity] \textit{line with numerical annotations} Depicts cardinality of one class in relation to another.
    \item[Aggregation] \textit{line with numerical annotations and diamond near parent} When the primary purpose of the parent is to contain the children.
    \item[Composition] \textit{line with shaded diamond near parent} Like aggregation, except the life cycle of the child relies is linked to the life cycle of the parent.
    \item[Inheritance] \textit{hollow arrow} Indicates that the child class is a specific type of the parent class.
    \item[Realization] \textit{hollow, dashed arrow}
\end{description}

\subsection{State Machines}
Two different levels:
\begin{description}
    \item[Protocol] legal usage scenarios.
    \item[Behavioural] individual entities.
\end{description}

State can either be:
\begin{description}
    \item[Qualitative] indicates current status of an object during its lifespan.
    \item[Quantitative] current values of all an object's attributes.
\end{description}

\section{Safety Critical Systems}
\begin{itemize}
    \item Reduce risk \textsc{so far as is reasonably practical}.
    \item Identify all possible risk reduction measures, and make determination of practicability.
    \item Make everything as simple as possible, but no simpler.
    \item Aim for quality during development, through the application of an established and recognised software process.
\end{itemize}

\subsubsection{Causes of faults}
\begin{description}
    \item[Coding Defects] implementation bugs. Memory errors, side effects, loss of precision, unchecked errors, etc.
    \item[Algorithmic Flaws] design logic. Wrong value calculated, incorrect value stored, unintended (de)activation.
    \item[Task Interference] race conditions, data corruption.
    \item[Insufficient Fault Protection] memory corruption, CPU overload, missed time-bounds, etc.
\end{description}

\subsection{Reviews}
\begin{description}
    \item[Technical Review] Review for conformance to standards, lead by team leader.
    \item[Software Inspection] Systemic location and classification of defects. Performed by people with at least as much experience as the software author.
    \item[Informal Inspection] Structured walk through of the work you've done with other people.
    \item[Audit] External, independently managed review of work. Done late in the development process.
\end{description}

\end{multicols}
\end{document}
